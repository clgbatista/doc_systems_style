\section{Protocolo}

Esta seção se dispõe a formalizar as leis de formação do protocolo de comunicação do subsistema de TT\&C com o segmento solo.

\subsection{Space Packet}

O \textit{Space Packet} é a mensagem a ser transferida entre os segmentos espacial e solo.
A seguir, sua lei de formação para codificação e decodificação.

\begin{table}[H]
\centering
\begin{tabular}{|c|c|}
\hline
\multicolumn{2}{|c|}{SPACE PACKET} \\ \hline
AX 42 header & DATA PACKET \\ \hline
\hl{17 B} & 0-255 B \\ \hline
\end{tabular}%
\end{table}

\begin{description}[align=right]
    \item[AX42 head] Cabeçalho do protocolo solo-bordo AX.42
    \item[Data Pckt] Informação a ser transmitida \textbf{máx. 255 B}
\end{description}

\begin{table}[H]
    \centering
    \begin{tabular}{|c|p{0.7\textwidth}|}
        \hline
        \rowcolor{orange}
        \textbf{RQ-TTC-311} & \textbf{O satélite deve ser capaz de codificar o \textit{Space Packet}} \\ \hline
    \end{tabular}
    \label{tab:rq-ttc-311}
\end{table}

\begin{table}[H]
    \centering
    \begin{tabular}{|c|p{0.7\textwidth}|}
        \hline
        \rowcolor{orange}
        \textbf{RQ-TTC-312} & \textbf{O satélite deve ser capaz de decodificar o \textit{Space Packet}} \\ \hline
    \end{tabular}
    \label{tab:rq-ttc-312}
\end{table}

\subsection{AX.42 Protocol}

O Protocolo AX.42 é o protocolo de comunicação solo-bordo, fictício, baseado no protocolo solo-bordo, real, rádio-amador AX.25.
A seguir, sua lei de formação para codificação e decodificação.

\begin{table}[H]
\centering
\begin{tabular}{|c|c|c|c|}
\hline
\multicolumn{4}{|c|}{AX42 Header} \\ \hline
Flag & Endereço & Control & PID \\ \hline
\hl{1 B} & 14 B & 1 B & 1 B \\ \hline
\end{tabular}%
\end{table}

\begin{description}[align=right]
    \item[Flag] Uma seqüência única de bits para indicar o início do frame \textbf{0x2A}
    \item[Endereço] Campo que especifica o endereço de destino e fonte.
    \item[Control] Comunicação do status do AX.42 \textbf{0x03}
    \item[PID] Identificador do Protocolo \textbf{0xF0}
\end{description}

\begin{table}[H]
    \centering
    \begin{tabular}{|c|p{0.7\textwidth}|}
        \hline
        \rowcolor{orange}
        \textbf{RQ-TTC-321} & \textbf{O satélite deve ser capaz de codificar AX42 \textit{Header}} \\ \hline
    \end{tabular}
    \label{tab:rq-ttc-321}
\end{table}

\begin{table}[H]
    \centering
    \begin{tabular}{|c|p{0.7\textwidth}|}
        \hline
        \rowcolor{orange}
        \textbf{RQ-TTC-322} & \textbf{O satélite deve ser capaz de decodificar AX42 \textit{Header}} \\ \hline
    \end{tabular}
    \label{tab:rq-ttc-322}
\end{table}

O campo de endereço é dividido da seguinte forma:

\begin{table}[H]
\centering
\begin{tabular}{|c|c|c|c|}
\hline
\multicolumn{4}{|c|}{Endereço} \\ \hline
Fonte & SSID & Destino & SSID \\ \hline
6 B & 1 B & 6 B & 1 B \\ \hline
\end{tabular}%
\end{table}

\begin{description}[align=right]
    \item[Fonte] Endereço da fonte composto por uma sequência alfanumérica, em caixa alta, de caracteres ASCII, apenas.
    \item[Destino] Endereço do destino composto por uma sequência alfanumérica, em caixa alta, de caracteres ASCII, apenas.
    \item[SSID] Byte numérico da fonte ou destino que identifica múltiplas estações ou satélites que utilizam o mesmo endereço \textbf{U8}.
\end{description}

O endereço da estação solo para o Cube Design é \textbf{CUBED0}, SSID \textbf{0}. 

\begin{table}[H]
    \centering
    \begin{tabular}{|c|p{0.7\textwidth}|}
        \hline
        \rowcolor{orange}
        \textbf{RQ-TTC-323} & \textbf{O endereço do satélite deve ser composto por seis caracteres ASCII, em caixa alta, e um byte identificador, SSID, numérico.} \\ \hline
    \end{tabular}
    \label{tab:rq-ttc-323}
\end{table}

\subsection{Data Packet}

O \textit{Data Packet} é o conjunto de dados a serem transmitidos via protocolo solo-bordo. Aqui há a identificação do telecomando ou telemetria, seu comprimento e parâmetros.
A seguir, sua lei de formação para codificação e decodificação.

\begin{table}[H]
\centering
\begin{tabular}{|c|c|c|c|}
\hline
\multicolumn{4}{|c|}{DATA Packet} \\ \hline
\multicolumn{3}{|c|}{Header} & \multirow{2}{*}{Info} \\ \cline{1-3}
Code & SubCode & Length & \\ \hline
1 B & 1 B & \hl{2 B} & 0-251 B \\ \hline
\end{tabular}
\end{table}

\begin{description}[align=right]
    \item[Header] Cabeçalho do Pacote de Dados.
    \item[Code] Identificador do Serviço.
    \item[SubCode] Identificador de Subserviço.
    \item[Length] Tamanho do frame do Data Packet, \textbf{Info}.
    \item[Info] Mensagem a ser transmitida.
\end{description}

O \textit{Info} corresponde aos campos \#4-5 dos \textit{frames} de telemetria e telecomando.

\begin{table}[H]
    \centering
    \begin{tabular}{|c|p{0.7\textwidth}|}
        \hline
        \rowcolor{orange}
        \textbf{RQ-TTC-331} & \textbf{O satélite deve ser capaz de codificar o \textit{Data Packet}} \\ \hline
    \end{tabular}
    \label{tab:rq-ttc-331}
\end{table}

\begin{table}[H]
    \centering
    \begin{tabular}{|c|p{0.7\textwidth}|}
        \hline
        \rowcolor{orange}
        \textbf{RQ-TTC-332} & \textbf{O satélite deve ser capaz de decodificar o \textit{Data Packet}} \\ \hline
    \end{tabular}
    \label{tab:rq-ttc-332}
\end{table}